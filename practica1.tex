\documentclass[12pt,a4paper]{article}

\usepackage[utf8]{inputenc}
\usepackage{graphicx}
\usepackage[spanish]{babel}
\usepackage{float}				%Para poner las imagenes exactamente donde se me cante las pelotas en caso de quererlo, poniendole [H]
\usepackage{amsmath}
\usepackage{epstopdf}
\usepackage{geometry}
\usepackage{hieroglf}
\usepackage{subcaption}
\usepackage[justification=centering]{caption}
\usepackage[colorlinks=true, allcolors=blue]{hyperref}
\geometry{
a4paper,
left=20mm,
right=20mm,
top=25mm,
bottom = 20mm
}
\usepackage{float}
\usepackage{units}
\marginparwidth=2cm
\usepackage[colorinlistoftodos]{todonotes}

% \usepackage{hyperref}   %Esto es para ir a los links

\title{\mathbf{Elementos Finitos \\Práctica 1}}
\author{Universidad de Cuenca}
\begin{document}
\maketitle
\begin{enumerate}
    \item Considerar el problema visto en clases:
    \begin{equation}
        \int^l_0\frac{dw}{dx}EA\frac{du}{dx}dx-\int^l_0wbdx=0
    \end{equation}
    con las condiciones de Dirichlet y asumiendo que tanto EA como b son constantes. Resolver empleando el método de elementos finitos con 3 elementos (4 nodos) de dos maneras:
    \begin{enumerate}
        \item Obteniendo directamente la matriz de rigidez global y el vector de fuerza global.
        \item Usando las matrices y vectores elementales y ensamblar.
\end{enumerate}
\end{document}